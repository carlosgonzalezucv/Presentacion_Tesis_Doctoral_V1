
\documentclass{bredelebeamer}
\usepackage{config}
%%%%%%%%%%%%%%%%%%%%%%%%%%%%%%%%%%%%%%%%%%%%%%%%

\title[Teoremas de tipo Bernstein--Doetsch]{
  Teoremas de tipo Bernstein--Doetsch para multifunciones 
  con terminos de error de tipo Takagi--Tabor.
}
% Titre du diaporama

\subtitle{Tesis Doctoral.}
% Sous-titre optionnel

\author [Carlos L. González.]{ 
  Autor: Carlos L. González. (UCV)\\
  Tutor: Dr. Nelson J. Merentes. (UCV) \\
  Co-tutor: Dr. Zsolt Páles. (UNIDEB)
}
% La commande \inst{...} Permet d'afficher l' affiliation de l'intervenant.
% Si il y a plusieurs intervenants: Marcel Dupont\inst{1}, Roger Durand\inst{2}
% Il suffit alors d'ajouter un autre institut sur le modèle ci-dessous.

%\today
% Optionnel. La date, généralement celle du jour de la conférence

\subject{math analysis convexity set-valued maps takagi transformation}
% C'est utilisé dans les métadonnes du PDF


\logo{
  \includegraphics[scale=0.075]{images/SelloUCV.png}
}

%%%%%%%%%%%%%%%%%%%%%%%%%%%%%%%%%%%%%%%%%%%%%%%%%%%%%%%%%%%%%%%%%%%%%
\begin{document}

\begin{frame}
  \titlepage
\end{frame}

\begin{frame}{Contenido}
  \tableofcontents
  % possibilité d'ajouter l'option [pausesections]
\end{frame}

\section{Antecedentes.}
\section{Antecedentes}

\subsection{Convexidad.}
\begin{frame}{El concepto de convexidad.}
  \Defi{*}{
    Sea $n\in\N$ y $H\subseteq\R^n$ un conjunto no vacío.
    Decimos que el conjunto $H$ es \emph{convexo\footnotemark}, si \alert{para todo}
    par de elementos $x,y \in H$, se tiene que el segmento
    de recta que los une
    \Eq{segment}{
      [x,y] := \{tx + (1-t)y \quad |\quad t\in[0,1]\}
    }
    est\'a, contenido en $H$, i.e., $[x,y]\subseteq H$.
  }
  \begin{figure}
    \centering
    \includegraphics[scale=0.3]{images/davinciplatonic.jpg}
    \includegraphics[scale=0.3]{images/convexity_by_archimides.png}
    \caption{Cinco s\'olidos plat\'onicos y la convexidad según Arquímides.}
  \end{figure}
  \footnotetext[1]{
    Hay en un plano ciertas líneas,
    que o bien se encuentran totalmente en el mismo lado 
    de las líneas rectas que unen sus extremidades,
    O no tienen parte de ellos en el otro lado.
    Archimedes of Syracuse (Sicily), ca 287 - ca 212 B.C.
    On the sphere and cylinder \alert{(Incluir referencia)}
  }
\end{frame} % Conjunto convexo
\begin{frame}{Epígrafo de una función.}
  \Defi{*}{
    Sea $f: H\subseteq\R^n\to\R$ una funci\'on. El \emph{epígrafo} de $f$
    es el conjunto
    \Eq{def_epigraph}{
      \Epi{f} := \{(x,y)\in H\times\R \quad|\quad y \geq f(x)\}
    }
  }
  \Defi{*}{
    Sea $H\subseteq\R^n$ un conjunto convexo y 
    sea $f: H\subseteq\R^n\to\R$ una funci\'on. 
    Decimos que $f$ es \emph{convexa}
    en $H$ si, $\Epi{f}$ es un conjunto convexo.
  }
  \Rem{*}{
    Sean $x,y\in H$. Note que $(x,f(x)), (y,f(y)) \in\Epi{f}$. Por lo tanto, si 
    $\Epi{f}$ es un conjunto convexo, entonces
    \Eq{*}{
        t(x,f(x)) &+ (1-t)(y,f(y)) 
          = (tx + (1-t)y, tf(x) + (1-t)f(y)) \in \Epi{f} \quad 
          \forall t\in[0,1], \\
        \mbox{i.e., }& \qquad 
            \fbox{ 
                $f(tx + (1-t)y) \leq tf(x) + (1-t)f(y))  
                    \quad\forall t\in[0,1],$ 
            }
    }
  }
  
\end{frame} % Epgirafo de una función y definicion de funcion convexa
\begin{frame}{Una equivalencia.}
  \Prp{*}{
    Sea $H\subseteq\R^n$ un conjunto convexo y 
    $f: H\subseteq\R^n\to\R$ una funci\'on. 
    $f$ es \emph{convexa} en $H$ sii, para todo $x,y\in H$ 
    se tiene que 
    \Eq{ytr}{
      f(tx + (1-t)y) \leq tf(x) + (1-t)f(y), \quad t\in[0,1].
    }
  }
\end{frame}
 % Una equivalencia para la convexidad de f


\section{Multifunciones.}
\begin{frame}{Multifunciones.}
	Sean $X$ y $Y$ espacios topol\'ogicos lineales y sea $D\subseteq X$
	un subconjunto abierto y convexo. Denotemos por $\P_0(Y)$ a la clase
	de subconjuntos no-vac\'ios de $Y$.
	\Defi{*}{
    	Una multifunci\'on $F:D\to\P_0(Y)$ es midconvexa en $D,$ si para todo
    	$x,y\in D$ se satisface 
    	\Eq{*}{
            \cmcvx{F(x)}{F(y)} \subseteq F\bigg(\cmcvx{x}{y}\bigg).
    	}
	}
	\Defi{*}{
    	Una multifunci\'on $F:D\to\P_0(Y)$ es midc\'oncava en $D,$ si para todo
    	$x,y\in D$ se satisface 
    	\Eq{*}{
    	    F\bigg(\cmcvx{x}{y}\bigg)\subseteq\cmcvx{F(x)}{F(y)}.
    	}
	}
\end{frame} % multifunciones

\section{Referencias}
\begin{frame}[allowframebreaks]
  \frametitle{Referencias}
  \scriptsize{\bibliographystyle{amsalpha}}%amsalpha
  \bibliography{publ,funcequ}
  \beamertemplatearticlebibitems
\end{frame}

\end{document}
