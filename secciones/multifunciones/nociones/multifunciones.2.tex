\begin{frame}{Un ejemplo.}
  \begin{exampleblock}{Una función midcóncava que no es midconvexa.}
    Sea $H$ un subconjunto no-vac\'io de $Y$. Definamos a la multifunci\'on $F$ mediante 
    la f\'ormula $F(t) = tH$, $t\in \R_+$. Para $t_1,t_2\in\R_+$ tenemos lo siguiente
    \Eq{*}{
      F\bigg(\frac{t_1+t_2}2\bigg) &= \frac{t_1+t_2}2H = \frac12(t_1+t_2)H \\ 
      &\subseteq \frac12(t_1H + t_2H) = \frac{t_1H + t_2H}2
      =\frac{F(t_1)+F(t_2)}2.
    }
    Por lo tanto, $F$ es una multifunci\'on c\'oncava. Sin embargo, la multifunción, 
    $G(t):=-F(t) = t(-H)$ no es midconvexa:
    \Eq{*}{
      \frac{G(t_1)+G(t_2)}2 = -\frac{t_1H + t_2H}2 \,\cancel{\subseteq}\, -\frac{t_1+t_2}2H = G\bigg(\frac{t_1+t_2}2\bigg).
    }
  \end{exampleblock}
\end{frame}