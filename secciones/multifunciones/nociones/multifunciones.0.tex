\begin{frame}{Multifunciones}
  Sean $X, Y$ espacios topológicos lineales y sea 
  $H\subseteq X$ un conjunto abierto y convexo. Denotaremos 
  por $\P_0(Y)$ a la clase de subconjuntos no-vacíos 
  de $Y$.
  \Defi{*}{
    Una \emph{multifunción} $F: H \to \P_0(Y)$ es 
    una aplicación que a cada elemento $x\in H$,
    le asocia un subconjunto no vacío $F(x)\subseteq Y$.
  }
  \begin{exampleblock}{Ejemplos}
    \begin{itemize}
      \item $S: X\to\P_0(Y)$ definida por $S(x) = \{x\}, \quad x\in X.$
      \item $F: \C\to\P_0(\C)$ definida por $F(z) = \sqrt[n]{z}, \quad z\in\C.$
      \item $H\subseteq Y$, $G: \R\to \P_0(Y)$ definida por $G(t) = tH, \quad t\in\R.$
    \end{itemize}
  \end{exampleblock}
    
\end{frame}