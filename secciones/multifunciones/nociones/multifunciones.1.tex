\begin{frame}{Midconvexidad.}
  Sea $F:H\to\P_0(Y)$ una multifunción.
  \Defi{*}{
    $F$ es midconvexa en $H,$ si para todo
    $x,y\in H$ se satisface 
    \Eq{*}{
      \cmcvx{F(x)}{F(y)} \subseteq F\bigg(\cmcvx{x}{y}\bigg).
    }
  }
  \Defi{*}{
    $F$ es midcóncava en $D,$ si para todo
    $x,y\in H$ se satisface 
    \Eq{*}{
      F\bigg(\cmcvx{x}{y}\bigg)\subseteq\cmcvx{F(x)}{F(y)}.
    }
  }
  \begin{alertblock}{Observación}
    \begin{center}
      $F$ es midconvexa $\xcancel{\iff}$ $-F$ es midcóncava.
    \end{center}
    %Este ejemplo evidencia que no existe hay una analog\'ia directa entre 
    %las multifunciones mid-convexas y las multifunciones mid-c\'oncavas. En 
    %consecuencia, los resultados obtenidos con respecto a estas clases de 
    %multifunciones deben tratarse por separado.
  \end{alertblock}
\end{frame}