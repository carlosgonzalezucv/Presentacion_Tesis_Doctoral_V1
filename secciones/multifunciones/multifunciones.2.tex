\begin{frame}{Multifunciones.}
	\Exa{*}{
    Sea $H$ un subconjunto no-vac\'io de $Y$. Definamos a la multifunci\'on $F$ mediante 
    la f\'ormula $F(t) = tH$, $t\in \R_+$. Para $t_1,t_2\in\R_+$ tenemos lo siguiente
    \Eq{*}{
      F\bigg(\frac{t_1+t_2}2\bigg) &= \frac{t_1+t_2}2H = \frac12(t_1+t_2)H \\ 
      &\subseteq \frac12(t_1H + t_2H) = \frac{t_1H + t_2H}2
      =\frac{F(t_1)+F(t_2)}2.
    }
    Por lo tanto, $F$ es una multifunci\'on c\'oncava. Sin embargo, la multifunción, 
    $G(t):=-F(t) = t(-H)$ no es midconvexa.
  }
  \Rem{*}{
    \begin{center}
      $F$ es midconvexa $\xcancel{\iff}$ $-F$ es midc\'oncava.
    \end{center}
    %Este ejemplo evidencia que no existe hay una analog\'ia directa entre 
    %las multifunciones mid-convexas y las multifunciones mid-c\'oncavas. En 
    %consecuencia, los resultados obtenidos con respecto a estas clases de 
    %multifunciones deben tratarse por separado.
	}
\end{frame}