\begin{frame}{Convexidad fuerte}
    \DefiOp{*}{
    	Sea $c>0$. Se dice que una función $f:D\to\R$ 
    	es fuertemente midconvexa con módulo $c$, si 	
    	para todo $x,y\in D$ y $t\in[0,1]$
    	\Eq{strgMidPol}{
    	    f\bigg(\frac{x+y}{2}\bigg)\leq \frac{f(x)+f(y)}2-\frac{c}4\|x-y\|^2,
    	}
	}{\cite{Pol66}}
	\Thm{*}{
    	Sea $c>0$. Si $f:D\to\R$ es una función fuertemente midconvexa 
    	con módulo $c$ y acotada superiormente en un subconjunto de $D$ 
    	con interior no vacío, entonces, $f$ es una función continua
    	y además fuertemente convexa con módulo $c$, i.e.,
    	\Eq{strgPol}{
    	    f(tx+(1-t)y)\leq tf(x)+(1-t)f(y)-ct(1-t)\|x-y\|^2,
    	}
	    para todo $x,y\in D,$ $t\in[0,1]$.
	}{\cite{AzoGimNikSan11}}
\end{frame}