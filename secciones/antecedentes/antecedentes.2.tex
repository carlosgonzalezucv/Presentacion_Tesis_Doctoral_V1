\begin{frame}{Epígrafo de una función.}
  \Defi{*}{
    Sea $f: H\subseteq\R^n\to\R$ una funci\'on. El \emph{epígrafo} de $f$
    es el conjunto
    \Eq{def_epigraph}{
      \Epi{f} := \{(x,y)\in H\times\R \quad|\quad y \geq f(x)\}
    }
  }
  \Defi{*}{
    Sea $H\subseteq\R^n$ un conjunto convexo y 
    sea $f: H\subseteq\R^n\to\R$ una funci\'on. 
    Decimos que $f$ es \emph{convexa}
    en $H$ si, $\Epi{f}$ es un conjunto convexo.
  }
  \Rem{*}{
    Sean $x,y\in H$. Note que $(x,f(x)), (y,f(y)) \in\Epi{f}$. Por lo tanto, si 
    $\Epi{f}$ es un conjunto convexo, entonces
    \Eq{*}{
        t(x,f(x)) &+ (1-t)(y,f(y)) 
          = (tx + (1-t)y, tf(x) + (1-t)f(y)) \in \Epi{f} \quad 
          \forall t\in[0,1], \\
        \mbox{i.e., }& \qquad 
            \fbox{ 
                $f(tx + (1-t)y) \leq tf(x) + (1-t)f(y))  
                    \quad\forall t\in[0,1],$ 
            }
    }
  }
  
\end{frame}