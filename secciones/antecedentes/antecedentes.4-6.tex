\begin{frame}{Función midconvexa y $\Q$-convexidad.}
  \Defi{*}{
    Diremos que la función $f$ es \emph{midconvexa}, si la 
    ecuación \eq{cvx} se cumple para $t=\frac12.$ Es
    decir, si para todo $x,y\in H$ se tiene que 
    \Eq{midcvx}{
        f\bigg(\frac{x+y}{2}\bigg)\leq\frac{f(x) + f(y)}2.
    }
  }
  \Defi{*}{
    Diremos que la función $f$ es \emph{$\Q$-convexa}, 
    si para todo $x,y\in H$ se tiene que 
    \Eq{qcvx}{
        f(tx + (1-t)y) \leq tf(x) + (1-t)f(y), \quad t\in[0,1]\cap\Q.
    }
  }
\end{frame}

\begin{frame}{Algunas observaciones.}
    \begin{exampleblock}{Observación}
        La desigualdad 
        \Eq{*}{
            f\bigg(\frac{x+y}{2}\bigg)\leq\frac{f(x) + f(y)}2.
        }
        implica que para todo $n\in\NN, x_1,x_2, \ldots, x_n\in H$
        \Eq{*}{
            f\bigg(\frac{x_1 + x_2 + \cdots + x_n}{n}\bigg)\leq\frac{f(x_1) + f(x_2) + \cdots + f(x_n)}n.
        }
        y a su vez, esto último implica que si $p,q\in\NN$, $q\neq0$, entonces 
        \Eq{*}{
            f\bigg(\frac{px + (q-p)y}{q}\bigg)\leq\frac{pf(x) + (q-p)f(y)}q.
        }
    \end{exampleblock}
    \Thm{*}{
        $f$ es midconvexa en $H$, si y sólo si, $f$ es $\Q$-convexa en $H$.
    }{Referencia}
\end{frame}

\subsection{El teorema de Bernstein--Doetsch.}
\begin{frame}{Teoremas de tipo Bernstein--Doetsch.}
    \Thm{*}{
        Toda función midconvexa y continua en $H$ es convexa.
    }{\cite{Kuc85}, Theorem 5.3.5.}
    \Thm{*}{
        Toda función midconvexa y localmente acotada superior
        en un punto, es convexa.
    }{\cite{Kuc85}, Theorem 6.4.2, \cite{BerDoe15}}
    \startchronology[startyear=1984,stopyear=2018,color=blue!50!white,arrow=false,dateselevation=5pt,height=5pt]
	
	\chronoevent[date=false,markdepth=70pt]{2017}{\tiny{\cite{GilGonNikPal15}}}
	\chronoevent[date=false,markdepth=60pt]{2014}{\tiny{\cite{GonNikPalRoa14}}}
	\chronoevent[date=false,markdepth=40pt]{2013}{\tiny{\cite{LeiMerNikSan13}}}
	\chronoevent[date=false,markdepth=25pt,ifcolorbox=false]{2012}{\tiny{\cite{MakPal12a}}}
	\chronoevent[date=false,markdepth=32pt,mark=false,ifcolorbox=false]{2012}{\tiny{\cite{Mur12DM}}}
	\chronoevent[date=false,markdepth=60pt]{2011}{\tiny{\cite{AzoGimNikSan11}}}
	\chronoevent[date=false,markdepth=68pt,mark=false,ifcolorbox=false]{2011}{\tiny{\cite{Haz11}}}
	\chronoevent[date=false,markdepth=73pt,mark=false,ifcolorbox=false]{2011}{\tiny{\cite{GilTro11}}}
	
	\chronoevent[date=false,ifcolorbox=false,markdepth=40pt]{2010}{\tiny{\cite{HazBur10}}}
	\chronoevent[date=false,ifcolorbox=false,markdepth=45pt]{2010}{\tiny{\cite{Haz10}}}
	\chronoevent[date=false,ifcolorbox=false]{2009}{\tiny{\cite{TabTab09a}}}
	\chronoevent[date=false,markdepth=18pt,mark=false,ifcolorbox=false]{2009}{\tiny{\cite{BurHazJuh09}}}
	\chronoevent[date=false,markdepth=25pt,mark=false,ifcolorbox=false]{2009}{\tiny{\cite{Haz09a}}}
	\chronoevent[date=false]{2005}{\tiny{\cite{HazPal05}}}
	\chronoevent[date=false,markdepth=40pt]{2004}{\tiny{\cite{HazPal04}}}
	\chronoevent[date=false,markdepth=50pt,mark=false]{2004}{\tiny{\cite{GilNikPal04}}}
	\chronoevent[date=false,markdepth=22pt,ifcolorbox=false]{2003}{\tiny{\cite{Nik03}}}
	\chronoevent[date=false,markdepth=30pt]{2001}{\tiny{\cite{Nik01}}}
	\chronoevent[date=false,markdepth=40pt,mark=false]{2001}{\tiny{\cite{NikPal01}}}
	\chronoevent[date=false,ifcolorbox=false]{2000}{\tiny{\cite{Pal00b}}}
	
	\chronoevent[date=false,markdepth=30pt]{1999}{\tiny{\cite{ChadMir99JST}}}
	\chronoevent[date=false]{1993}{\tiny{\cite{NgNik93}}}
	\chronoevent[date=false,markdepth=40pt]{1991}{\tiny{\cite{KomKuc91b}}}
	
	\chronoevent[date=false,markdepth=20pt,mark=false]{1989}{\tiny{\cite{Nik89}}}
	\chronoevent[date=false]{1989}{\tiny{[KomKuc89]\nocite{KomKuc89b}}}
	\chronoevent[date=false,markdepth=40pt]{1987}{\tiny{\cite{Nik87a}}}
	\chronoevent[date=false,ifcolorbox=false,markdepth=50pt,mark=false]{1987}{\tiny{\cite{Nik87c}}}
	\chronoevent[date=false,ifcolorbox=false]{1986}{\tiny{\cite{Nik86}}}
	\chronoevent[date=false,ifcolorbox=false]{1984}{\tiny{\cite{Tru84}}}
	
\stopchronology
\end{frame}