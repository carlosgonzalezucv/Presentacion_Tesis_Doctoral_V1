\section{Antecedentes}

\subsection{Convexidad.}
\begin{frame}{El concepto de convexidad.}
    
  \Defi{*}{
    Sea $n\in\N$ y $H\subseteq\R^n$ un conjunto no vacío.
    Decimos que el conjunto $H$ es \emph{convexo}, si \alert{para todo}
    par de elementos $x,y \in H$, se tiene que el segmento
    de recta que los une
    \Eq{segment}{
      [x,y] := \{tx + (1-t)y \quad |\quad t\in[0,1]\}
    }
    est\'a, contenido en $H$, i.e., $[x,y]\subseteq H$.
  }
  \Defi{*}{
    Sea $f: [a,b]\to\R$ una funci\'on. Decimos que $f$ es \emph{convexa}
    en $[a,b]$ sii, para todo $x,y\in[a,b]$ se tiene que 
    \Eq{ytr}{
      f(tx + (1-t)y) \leq tf(x) + (1-t)f(y), \quad t\in[0,1].
    }
  }
\end{frame}